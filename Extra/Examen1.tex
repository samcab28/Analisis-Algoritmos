\documentclass{article}
\usepackage{amsmath}
\usepackage{titlesec}
\usepackage{geometry}
\usepackage{fancyhdr}
\usepackage{listings}
\usepackage[spanish]{babel}
\usepackage[utf8]{inputenc}
\usepackage{xcolor}
\usepackage{blindtext}

\title{Examen I EXTRA}
\author{Mynell Myers Hall}
\date{\today}

\lstdefinestyle{customcpp}{
    language=C++,
    basicstyle=\ttfamily\small,
    keywordstyle=\color{cyan!50},
    stringstyle=\color{green!50!black},
    numbers=left,
    numberstyle=\tiny\color{gray},
    frame=none,
    backgroundcolor=\color{gray!10},
    rulecolor=\color{gray!10},
    identifierstyle=\color{purple!40},
}

\begin{document}
\maketitle

\section*{Pregunta 1}

\subsection*{(a)}
\[
T(n) = 3 \cdot T\left(\frac{n}{3}\right) + 5n^2 + 3n
\]
\[
a = 3, \quad b = 3, \quad f(n) = 5n^2 + 3n
\]
Usando el Teorema Maestro:
\[
T(n) = \Theta(n^{\log_b a} \cdot \log^k n)
\]
\[
T(n) = \Theta(n^{\log_3 3} \cdot \log^0 n)
\]
\[
T(n) = \Theta(n^2)
\]

\subsection*{(b)}
\[
T(n) = 5 \cdot T\left(\frac{n}{4}\right) + 4n
\]
\[
a = 5, \quad b = 4, \quad f(n) = 4n
\]
Usando el Teorema Maestro:
\[
T(n) = \Theta(n^{\log_b a})
\]
\[
T(n) = \Theta(n^{\log_4 5})
\]

\subsection*{(c)}
\[
T(n) = T\left(\frac{n}{2}\right) + 2n
\]
\[
a = 1, \quad b = 2, \quad f(n) = 2n
\]
Usando el Teorema Maestro:
\[
T(n) = \Theta(f(n))
\]
\[
T(n) = \Theta(2n)
\]
\[
T(n) = \Theta(n)
\]

\subsection*{(d)}
\[
T(n) = 36 \cdot T\left(\frac{n}{6}\right) + \frac{n(n-1)}{2}
\]
\[
a = 36, \quad b = 6, \quad f(n) = \frac{n(n-1)}{2}
\]
Usando el Teorema Maestro:
\[
T(n) = \Theta(n^{\log_b a} \cdot \log^k n)
\]
\[
T(n) = \Theta(n^{\log_6 36} \cdot \log^0 n)
\]
\[
T(n) = \Theta(n^2)
\]

\section*{Pregunta 9}

\begin{lstlisting}[style=customcpp]
#include <iostream>
#include <vector>
#include <algorithm>
using namespace std;

int minOpPalindromo(const string& C) {
    int n = C.length();
    vector<vector<int>> memo(n, vector<int>(n, 0));
    for (int i = 0; i < n; ++i) {memo[i][i] = 0;}

    for (int cl = 2; cl<= n;cl++) {
        for (int i = 0; i <= n - cl; ++i) {
            int j= i +cl -1;
            if (C[i] == C[j]) {memo[i][j] = memo[i + 1][j - 1];
            } else {
                memo[i][j] = min({memo[i + 1][j], memo[i][j - 1], memo[i + 1][j - 1]}) + 1;
            }
        }
    }
    return memo[0][n - 1];
}
\end{lstlisting}

\section*{Pregunta 10}

\subsection*{Grafo Bipartito $G = (N, C)$}
\[ N = \{1, 2, 3, 4\} \]
\[ C = \{\{1, 3\}, \{2, 4\}, \{1, 2\}, \{3, 4\}\} \]

\subsection*{Matroide $M_1 = (E, I_1)$}
\[ E = C \]
\[ I_1 \text{ es el conjunto de subconjuntos que forman apareamientos en } G \]
\[
I_1 = \{
\{\{1, 3\}\}, \{\{2, 4\}\}, \{\{1, 2\}\}, \{\{3, 4\}\}, \\
\{\{1, 3\}, \{2, 4\}\}, \{\{1, 3\}, \{1, 2\}\}, \{\{1, 3\}, \{3, 4\}\}, \\
\{\{2, 4\}, \{1, 2\}\}, \{\{2, 4\}, \{3, 4\}\}, \{\{1, 2\}, \{3, 4\}\}, \\
\{\{1, 3\}, \{2, 4\}, \{1, 2\}\}, \{\{1, 3\}, \{2, 4\}, \{3, 4\}\}, \\
\{\{1, 2\}, \{3, 4\}, \{1, 3\}\}, \{\{1, 2\}, \{3, 4\}, \{2, 4\}\}, \\
\{\{1, 2\}, \{3, 4\}, \{1, 2\}\}, \{\{1, 2\}, \{3, 4\}, \{1, 3\}\}, \\
\{\{1, 2\}, \{3, 4\}, \{2, 4\}\}
\}
\]

\subsection*{Matroide $M_2 = (E, I_2)$}
\[ E = C \]
\[ I_2 \text{ es el conjunto de subconjuntos que también forman apareamientos en } G \]
\[
I_2 = \{
\{\{1, 3\}\}, \{\{2, 4\}\}, \{\{1, 2\}\}, \{\{3, 4\}\}, \\
\{\{2, 4\}, \{1, 3\}\}, \{\{1, 2\}, \{3, 4\}\}
\}
\]

\subsection*{Demostración de Propiedades}

\subsection*{1. Demostracion de matroides}

\subsubsection*{Matroide $M_1$}

Si $A \in I_1$ y $B \subseteq A$, entonces $B \in I_1$.\\

\textbf{Demostración:} Si $A$ es un apareamiento en $G$, entonces cualquier subconjunto de $A$ sigue siendo un apareamiento. Por lo tanto, $B \subseteq A$ también es un apareamiento en $G$.\\

Si $A \in I_1$, $B \in I_1$ y $|A| < |B|$, entonces existe algún elemento $x$ en $B - A$ tal que $A \cup \{x\} \in I_1$.\\

\textbf{Demostración:} Dado que $A$ y $B$ son dos apareamientos en $G$, podemos tomar cualquier elemento $x$ en $B - A$ y agregarlo a $A$ sin romper la propiedad de apareamiento.

\subsubsection*{Matroide $M_2$}

Si $A \in I_2$ y $B \subseteq A$, entonces $B \in I_2$.\\

\textbf{Demostración:} Similar a la demostración de $M_1$, cualquier subconjunto de un apareamiento en $G$ sigue siendo un apareamiento. Por lo tanto, $B \subseteq A$ también es un apareamiento en $G$.\\

Si $A \in I_2$, $B \in I_2$ y $|A| < |B|$, entonces existe algún elemento $x$ en $B - A$ tal que $A \cup \{x\} \in I_2$.\\

\textbf{Demostración:} Dado que $A$ y $B$ son dos apareamientos en $G$, podemos tomar cualquier elemento $x$ en $B - A$ y agregarlo a $B$ sin romper la propiedad de apareamiento.


\subsection*{3. Intersección que representa apareamientos}

$M_1 \cap M_2$ consiste en los conjuntos $\{\{1, 3\}\}, \{\{2, 4\}\}, \{\{1, 2\}\}, \{\{3, 4\}\}$, que son precisamente los apareamientos de $G$.\\

\textbf{Demostración:} Cualquier conjunto independiente en $M_1 \cap M_2$ es un conjunto independiente en ambos $M_1$ y $M_2$, lo que implica que es un apareamiento en $G$.

\end{document}
